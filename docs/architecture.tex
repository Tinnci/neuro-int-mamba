\documentclass[11pt,a4paper]{article}
\usepackage[utf8]{inputenc}
\usepackage{fontspec}
\usepackage{amsmath}
\usepackage{hyperref}
\usepackage{geometry}
\geometry{margin=1in}

\title{Neuro-INT Mamba: Architectural Principles}
\author{Neuro-INT Mamba Project}
\date{\today}

\begin{document}

\maketitle

\section{Introduction}
Neuro-INT Mamba is a bio-inspired architecture designed for dexterous manipulation. It integrates State Space Models (SSM) with key neurobiological principles to achieve real-time, adaptive, and multi-modal control.

\section{Core Architectural Pillars}

\subsection{Dual-Stream Intrinsic Neural Timescales (INT)}
The architecture implements parallel processing streams with different discretization steps ($\Delta$):
\begin{itemize}
    \item \textbf{Fast Stream (Sensory)}: Small $\Delta$, simulating the Primary Sensory Cortex (S1/M1) for rapid response to high-frequency stimuli.
    \item \textbf{Slow Stream (Cognitive)}: Large $\Delta$, simulating the Prefrontal Cortex for long-term integration and goal maintenance.
\end{itemize}

\subsection{Predictive Coding \& Efference Copy}
Inspired by the brain's ability to predict sensory consequences of motor commands:
\begin{itemize}
    \item Each layer generates a prediction of the next state.
    \item The network processes the \textit{prediction error} ($x_{actual} - x_{predicted}$), significantly reducing redundancy.
    \item \textbf{Intent Prior}: High-level intent (e.g., from EMG) acts as a prior to bias the prediction, enabling faster error correction during unexpected perturbations.
\end{itemize}

\subsection{Chandelier Gating Mechanism}
Mimics the inhibitory control of Chandelier cells (ChCs) on pyramidal neurons:
\begin{equation}
    Gate(x) = x \cdot \sigma(\alpha - \beta \cdot \|x\|^2)
\end{equation}
This mechanism prevents neural "over-firing" and ensures stability during high-intensity sensory input.

\subsection{Spinal Reflex Loop}
A low-level, fast feedback loop that operates in parallel with cortical processing. It implements a Proportional-Derivative (PD) control logic:
\begin{equation}
    u_{reflex} = -(K_p \cdot \theta + K_d \cdot \dot{\theta})
\end{equation}
where $\theta$ and $\dot{\theta}$ represent joint positions and velocities. This mimics muscle spindle sensitivity. The gains $K_p$ and $K_d$ can be dynamically modulated by higher-level intent (e.g., EMG intensity) to adjust joint stiffness during contact.

\subsection{Thalamic Multi-modal Fusion}
The Thalamic Encoder acts as a gateway, projecting diverse sensory modalities into a unified latent space:
\begin{itemize}
    \item \textbf{Tactile Spatial Prior}: Uses 1D Convolutional layers to capture spatial correlations in array-based tactile sensors.
    \item \textbf{High-Res Vision}: Optimized projection layers with LayerNorm for stable integration of high-dimensional visual features.
    \item \textbf{EMG Transfer Learning}: Integrates surface Electromyography (sEMG) signals to capture human motor intent.
    \item \textbf{Muscle Synergy Bottleneck}: Extracts low-dimensional control primitives (synergies) from high-dimensional intent, enabling human-like coordination.
\end{itemize}
This enables coherent cross-modal integration before cortical processing.

\section{Real-time Simultaneous I/O}
Unlike traditional Transformers that suffer from $O(L^2)$ or $O(L)$ inference latency, Neuro-INT Mamba leverages the recurrent form of SSMs to achieve $O(1)$ per-step inference, allowing for seamless input-output coupling in closed-loop robotic control.

\end{document}
